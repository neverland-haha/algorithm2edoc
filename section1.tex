% !TEX encoding = UTF-8

%\noindent \cprotect \fbox{|\SetAlgorithmName{algorithmname}{algorithmautorefname}{list of algorithms name}|}	\\


%第一节
%代码
\section{从一个实例开始}
\begin{lstlisting}
\begin{algorithm}[H]
	\SetAlgoLined
	\KwData{this text}
	\KwResult{how to write algorithm with \LaTeX2e }
	initialization\;
	\While{not at end of this document}{
		read current\;
		\eIf{understand}{
			go to next section\;
			current section becomes this one\;
		}{
			go back to the beginning of current section\;
		}
	}
	\caption{How to write algorithms}
\end{algorithm}
\end{lstlisting}

%代码的排版	
\begin{algorithm}[H]
	\SetAlgoLined
	\KwData{this text}
	\KwResult{how to write algorithm with \LaTeX2e }
	initialization\;
	\While{not at end of this document}{
		read current\;
		\eIf{understand}{
			go to next section\;
			current section becomes this one\;
		}{
			go back to the beginning of current section\;
		}
	}
	\caption{How to write algorithms}
\end{algorithm}

\vspace{0.5em}

\par \textbl{algorithm} 与 \textbl{figure} 和 \textbl{table} 一样,都属于浮动体,都需要在浮动体的
环境下进行排版,所以大多数与两种常规的环境相似,但在撰写此文档中也发现,倘若将上述代
码放入一个子文件中再利用 \verb|\include{file}| 后算法块会自动浮动到下一页的顶部,且这种矛盾
无法调和.
\par 在宏包的 doc 里这样解释到:可选参数 |[Hhtbp]| 的作用类似于图形环境. H 参数强制算
法保持不变.如果使用,则算法不再是浮动对象.注意:算法无法剪切,因此,如果在给定位置
没有足够的位置放置带有 H 选项的算法,则 \LaTeX 将放置一个空格并将该算法放在下一页.
\par \txr{需要注意的是},每一行的结尾必须以 |\;| 结束,只有那些以宏命令开始的不应该以 
|\;| 结束,例如在此实例中的 |\SetAlgoLined|、|\KwData{var}| 属于宏命令,不需要以 |\;| 结尾,而
例如 |\initialization| 这样的行内的命令需要以 |\;| 结束, 事实上可以理解为需要打出冒号而加入转义
字符.
\par 标题的工作方式与图形环境相同,不同之处在于标题应位于算法的末尾. |\listofalgorithms| 将
其用作算法列表的参考名称.还可以使用与软件包一起提供的 \textbl{title} 宏,但是此宏不会在算
法列表中插入一个条目.关于标题的设定,在后续的章节会有讲述.

%第二节
\section{常用的环境以及一些基本的修改}

%2.1节
\subsection{常用的环境}

\noindent \textcolor{pack}{algorithm2e} 提供 4 种环境:
\begin{description}
	\item[algorithm:] 这是最常用的环境。
	\item[algorithm*:]  与前者一样,但它用于两列文本中,使算法跨两列
	\item[procedure:]该环境类似于算法环境,但是:
	\begin{itemize}
		\item 推荐使用 |ruled| 和 |algoruled|.
		\item 标题写作 |Procedure name|.
		\item |\caption| 命令的语法限制如下:您必须输入名称,后接2个大括号,例如“ Name()”.
		您可以将参数放在大括号中,然后放在文本中.如果未提供任何参数,括号将在标题中删除.
		\item 现在,label 将 procedure 或 function 的名称(标题中大括号前的文本)作为参考(而不
		是经典算法环境中的数字).
		\item 标题中设置的 procedure 或 function 的名称会自动定义为 KwFunction,因此可以用作宏
		.例如,如果在过程环境中设置 |\caption{myproc()}|,则可以在主文本中使用 |\myproc| 宏
		.注意该宏仅在 |\caption| 之后定义!
		\item |nokwfunc| 无法在 function 和 procedure 环境中使用上述功能. 例如,如果您将不能使
		用命令名称的 procedure 或 function 的名称用作数学显示时它则很有用.
	\end{itemize}
	\item[procedure*:] 与上述的 procedure 一致,但它用于 two columns 模式中,使 procedure 跨
	两列.
	\item [function:]  与前面的一致,但标题中使用 Function 而不是 Procedure.
	\item [function:] 与前面的一致,但在 two columns 模式中使得文本跨两列.
\end{description}

%2.2节 
\subsection{一些基本关键词的修改}
\txr{如果你不喜欢原来的算法标题或寻找其他方法进行修改,可以使用以下命令更改算法的名称:}

\noindent \cprotect 
\fbox{|\SetAlgorithmName{algorithmname}{algorithmautorefname}{list of algorithms name}|}	

它将重新定义重新定义算法的名称和算法的句子列表.例如,我们想定义一个算法的名称为算法x
我们采用两种方式,|\SetAlgorithmName{算法}{算法}{ }| 及 
|\renewcommand{\algorithmcfname}{算法}{}{}|  	\\
其中 |<list of algorithms name>| 插入具有标题的所有算法的列表.

\noindent \cprotect 
\fbox{|\SetAlgoProcName{aname}{anautorefname}|}	

设置 procedure 的标题名称,其中第二个参数是 \textbl{hyppref} 宏包的 |\autoref| 将使用的名称.

\noindent \cprotect 
\fbox{|\SetAlgoFuncName{aname}{anautorefname}|}	

与 procedure、algorithm 的修改方法一致,同样的,第二个参数也是 \textbl{hyppref} 宏包的 
|\autoref| 将使用的名称.
%


\par 由于在 \textcolor{pack}{algorithm2e} 中已经预先定义好了一些关键词,例如 
|\KwIn{}|、|\KwOut{}| 以及 |\KwData{}| 等,我们可以采用如下方法进行重新定义,
即 |\SetKw{KwData}{\textcolor{red}{\textbf{变量和参数:}}}{}|,同理我们也可以对上述其他参数
进
行相应的设置.具体示例及所见效果见 \autoref{algo2}.

%\renewcommand{\algorithmcfname}{算法}{}{}		%修改caption名称可以
\SetKw{KwData}{\textcolor{red}{\textbf{变量和参数:}}}{}			%修改变量名称
\SetKwInput{KwOut}{输出}
\SetAlgorithmName{算法}{}{}									%修改caption这样也可以
\begin{algorithm}[htbp]
	\SetAlgoNoLine
	\KwData{this text}
	
	\KwResult{how to write algorithm with \LaTeX2e }
	initialization\;
	\While{not at end of this document}{
		read current\;
		\eIf{understand}{
			go to next section\;
			current section becomes this one\;
		}{
			go back to the beginning of current section\;
		}
	}
\KwOut{this is output}
	\caption{如何写一个算法}
	\label{algo2}
\end{algorithm}
 	
 \noindent 在 \textcolor{pack}{alogrithm2e} 中定义了如下的宏:
 	
 	\noindent 输入与输出
 	\begin{itemize}
 		\item |\KwIn{input}|
 		\item |\KwOut{output}|
 		\item |\KwData{input}|
 		\item |\KwResult{output}|
 	\end{itemize}
 	
 	\noindent 基本的关键词和块
 	\begin{itemize}
 		\item |\KwTo|
 		\item |\KwRet{[value]}|
 		\item |\Return{[value]}|
 	\end{itemize} 
 	\begin{itemize}
 		\item |\Begin{block inside}|
 		\item |\Begin(begin comment){block inside}|
 	\end{itemize}
 	\txr{需要注意的是,这些关键词是严格区分大小写的,所以需要确保用户的输入正确}.其他的关
 	键词也可以 \\ |texdoc algorithm2e| 进行查阅.
 
%2.3节
\subsection{控制算法的布局}
	\par 由于在引入宏包时,在 option 选项中加入了 |vlined| 以及 |ruled| 的参数,所以在伪代码中
	会有上
	下尺,并且在一定的伪代码中会出现垂直尺.
	\par  当然,外部的封装环境不止 ruled 一种选项,也可以选择完全封闭的选项亦或者完全没有
	外
	部边框的选项.内部提供的封装环境有以下几种选项:
	\begin{itemize}
		\item[boxed:] 将算法封装在框中.
		\item[boxruled:] 用方框将算法环绕,将标题放在上方,并在标题后添加一行.
		\item[ruled:] 在顶部和底部都有一条线的算法.请注意,标题不再位于算法下方,而是在算法
		开始时设置.
		\item[algoruled:] 如上所述,但在尺后留有多余的空格.
		\item[tworuled:] tworuled的行为就 ruled 的一样,但标题后面没有加一行.
		\item[plain:] 默认值,无功能.
	\end{itemize}
	
	
	\noindent \cprotect \fbox{|\RestyleAlgo{style}|} \\
由于在可选参数中加入了 ruled 选项,如果需要临时修改算法的样式的话,可以采用此命令临时修改某一个算法的外观布局.例如 \textit{boxed, boxruled, ruled and algoruled}.

\noindent \cprotect \fbox{|\SetAlgoVlined |} \\
此命令能够让你在每个块的开始和结束之间打印一条垂直线,然后打印一条水平线.

\noindent \cprotect \fbox{|\SetInd{before rule space}{after rule space} |} \\
设置垂直标尺之前和之后的空间大小.在 |\OnLine| 模式下,缩进空间是这两个值的和,默认情况
下为 0.5em 和 1em.

\noindent \cprotect \fbox{|\SetAlgoHangIndent {length}|}\\
  设置 \textit{hangindent} 用于在 \textit{long} 语句中的第一行之后缩进后继行的缩进长度值.

\noindent \cprotect \fbox{|\Setvlineskip{length}|}	\\ 
设置小水平线之后的垂直空间的值,该水平线在 \textit{vlined} 模式下关闭一个 block.

\noindent \cprotect \fbox{|\SetAlgoSkip{skip command}|} 	\\
  算法在前后放置额外的垂直空间(与文本之间的前后),以免出现框状或规则式算法的文本颠簸
行.默认情况下,这是
一个.您可以使用此宏更改此值.四种可能性是
\begin{itemize}
	\item 	|\SetAlgoSkip{}| 没有额外的垂直跳动.
	\item  |\SetAlgoSkip{smallskip}|  设置为默认的.
	\item |\SetAlgoSkip{medskip}|   有一个较大的间距.
	\item  | \SetAlgoSkip{bigskip}|  有一个更大的间距.
\end{itemize}

\noindent \cprotect \fbox{|\SetAlgoInsideSkip{skip command}|}  \\
算法前后没有多余的垂直空间算法的核心.因此,文本将以方框或格线样式放在行后.要放置多余的空间,需要使用 |\SetAlgoInsideSkip{skip command}| ,

\noindent \cprotect \fbox{|\IncMargin{length}|}  \\ 
增加算法文本到边界的距离.

\noindent \cprotect \fbox{|\DecMargin{length}|}  \\ 
减少算法文本到边界的距离.

\noindent \cprotect \fbox{|\setlength{\interspacetitleruled}{xx em}|}  \\
它控制的是 top rule 和 mid rule 和文本间的距离,默认的距离为 2pt.

\noindent \cprotect \fbox{|\setlength{\interspacetitleboxruled}{x \lineskip}|} \\
它控制盒式算法中规则和标题之间的垂直空间.默认的距离为 |2 \lineskip|.

\noindent \cprotect \fbox{|\SetCustomAlgoRuledWidth{length}|} \\
这个选项控制的是外部封装尺的长度,如果你的算法不够长可以考虑修改此选项.

\subsection{设置算法的标题和题目}
	\noindent \cprotect \fbox{|\SetAlgoCaptionSeparator{sep}|} \\
	这将在算法标题(算法1)和算法名称之间设置分隔符.默认情况下为  |“ : ”|,标题看起来像
	 |“algorithm3:名称”| ,但是现在您可以通过使用示例来更改它,该名称将为  
	 |“algorithm3.名称”|.
	 
	 \noindent \cprotect \fbox{|\AlCapSkip{}|} \\
	 是纯文本和盒装模式下算法主体与 caption 之间距离的尺寸.可以手动更改,也可以使用
	 |\SetAlCapSkip {0ex}| 进行更改,当然也可以使用 |\SetAlCapSkip{length}| 进行修改.
	 
	\noindent \cprotect \fbox{|\SetAlCapHSkip{length}|}\\
	在 ruled 选项中,caption 位于 toprule 和 mid rule 之间,可以通过设置此选项进行水平缩进.
	
	\txr{下面是 title 和 caption 的样式}:  
	\begin{description}
		\item|\AlCapSty{<text>}|: |sets <text>| 它与 |\AlCapFnt| 一起使用,以打印 |Algorithm #:| 更准
		确的说是\\ |\AlCapSty{\AlCapFnt Algorithm #:}|.
		\item  	|\AlCapNameSty{<text>}|  在caption 名称排版中设置 |<text>|,该名称与 
		|\AlCapNameFnt| 一起使用以	打印通过调用  |\caption {name}| 
		设置的标题名称.更准确地说,其打印出
		|\AlCapNameSty{\AlCapNameFnt 名称}|\\
		给出“caption”.默认情况下,|\AlCapNameSty|为 	|textnormal|
	\end{description}
	\txr{事实上,只要了解知道有 \textit{caption、title、procedre、function}这些 words 后在到文档中查找修改命令即可,更多样式的修改可以自行参阅文档,再此不再一一列举.}
	
\section{修改 rule 的颜色和厚度}
对于希望修改 ruled 环境中的 rule 的颜色
\href{https://tex.stackexchange.com/questions/298340/colored-horizontal-lines-in-algorithm2e/298346
#298346}{texstackexchange} 中提供了一种解决办法.  而想要修改 rule 厚度的尚有待考察,宏包内
没有直接修改的命令,需要修改原始命令.

	\begin{lstlisting}
\makeatletter
\newcommand{\setalgotoprulecolor}[1]{\colorlet{toprulecolor}{#1}}
\let\old@algocf@pre@ruled\@algocf@pre@ruled % Adjust top rule colour
\renewcommand{\@algocf@pre@ruled}{\textcolor{toprulecolor}{\old@algocf@pre@ruled}}

\newcommand{\setalgobotrulecolor}[1]{\colorlet{bottomrulecolor}{#1}}
\let\old@algocf@post@ruled\@algocf@post@ruled % Adjust middle rule colour
\renewcommand{\@algocf@post@ruled}{\textcolor{bottomrulecolor}{\old@algocf@post@ruled}}

\newcommand{\setalgomidrulecolor}[1]{\colorlet{midrulecolor}{#1}}
\renewcommand{\algocf@caption@ruled}{%
	\box\algocf@capbox{\color{midrulecolor}\kern\interspacetitleruled\hrule
		width\algocf@ruledwidth height\algotitleheightrule depth0pt\kern\interspacealgoruled}}
\makeatother

\setalgotoprulecolor{blue!30}% Default
\setalgobotrulecolor{red!30}% Default
\setalgomidrulecolor{green!30}% Default
	\end{lstlisting}

%修改颜色的样例
\makeatletter
\newcommand{\setalgotoprulecolor}[1]{\colorlet{toprulecolor}{#1}}
\let\old@algocf@pre@ruled\@algocf@pre@ruled % Adjust top rule colour
\renewcommand{\@algocf@pre@ruled}{\textcolor{toprulecolor}{\old@algocf@pre@ruled}}

\newcommand{\setalgobotrulecolor}[1]{\colorlet{bottomrulecolor}{#1}}
\let\old@algocf@post@ruled\@algocf@post@ruled % Adjust middle rule colour
\renewcommand{\@algocf@post@ruled}{\textcolor{bottomrulecolor}{\old@algocf@post@ruled}}

\newcommand{\setalgomidrulecolor}[1]{\colorlet{midrulecolor}{#1}}
\renewcommand{\algocf@caption@ruled}{%
	\box\algocf@capbox{\color{midrulecolor}\kern\interspacetitleruled\hrule
		width\algocf@ruledwidth height\algotitleheightrule depth0pt\kern\interspacealgoruled}}
\makeatother

\setalgotoprulecolor{blue!30}% Default
\setalgobotrulecolor{red!30}% Default
\setalgomidrulecolor{green!30}% Default


	{	
	\begin{algorithm}[htbp]
		\LinesNumbered
		\KwData{this text}
		
		\KwResult{how to write algorithm with \LaTeX2e }
		initialization\;
		\While{not at end of this document}{
			read current\;
			\eIf{understand}{
				go to next section\;
				current section becomes this one\;
			}{
				go back to the beginning of current section\;
			}
		}
		\KwOut{this is output}
		\caption{如何写一个彩色边框的算法}
		\label{algo4}
	\end{algorithm}
}

